\documentclass{beamer}
\usepackage{amsmath}
\input{Algo1Macros}

\title{Prueba de Oposición}
\author{Eric Brandwein}
\date{5 de Mayo de 2023}

\begin{document}

\frame{\titlepage}

\section{Marco}
\begin{frame}{Marco}
    Guías de ejercicios:
    \begin{enumerate}
        \item Lógica
        \item Introducción al Lenguaje de Especificación
        \item {\bfseries\color{blue}Especificación de problemas}
        \item Precondición más débil en SmallLang
        \item Demostración de corrección de ciclos en SmallLang
        \item Testing
        \item Ciclos a partir de invariantes
        \item Tiempo de Ejecución de Peor Caso de un Programa
    \end{enumerate}
\end{frame}

\begin{frame}{Guía 3: Especificación de Problemas}
    \large
    Con esta guía se busca que el estudiante obtenga las herramientas básicas para después poder demostrar que un algoritmo es correcto. Se trata principalmente sobre el \textbf{contrato} de un algoritmo.\pause
    \vspace{2em}
    \begin{itemize}
        \item Se introducen los conceptos de pre- y poscondición.\pause
        \item Se comienzan a implementar especificaciones de problemas con los \texttt{proc}. 
    \end{itemize}
\end{frame}

\section{Ejercicio}
\begin{frame}{Ejercicio}
    \large
    El siguiente ejercicio nos pide analizar un \texttt{proc} y determinar si es correcto o no, para familiarizarnos con el concepto de pre- y poscondición antes de comenzar a implementarlos nosotros mismos. 

\end{frame}

\begin{frame}{Ejercicio 2}
Veamos la siguiente especificación de un problema:
\begin{proc}{elementosQueSumen}{\\
    \qquad$\In l: \TLista{\ent}, \In suma: \ent, \Out result: \TLista{\ent}$\\
}{}{}
    \begin{adjustwidth}{+2em}{}   
        \pre{True}
    \end{adjustwidth}
    \begin{adjustwidth}{+2em}{}
        \post{
            contenida(result, l) \land suma = \sum_{i = 0}^{|result| - 1}result[i]
        }
    \end{adjustwidth}
\end{proc}

\pred{contenida}{$adentro: \TLista{\ent}, afuera: \TLista{\ent}$}{
    \begin{adjustwidth}{+2em}{}    
        $(\forall x: \ent)(
            \# apariciones(x,adentro) \leq \# apariciones(x,afuera)
        )$
    \end{adjustwidth}
}
\pause

\begin{enumerate}[a)]
    \item ¿Es válida esta especificación? En caso contrario, mostrar por qué.\pause
    \item ¿Qué tenemos que cambiar para que sea válida?
\end{enumerate}

\end{frame}

\begin{frame}
    \center\huge
    ¡Gracias!
\end{frame}
\end{document}